\documentclass[relatorio.tex]{subfiles}
% Load Map

\begin{document}
\section{Inicialização de um Mapa}
\subsection{Rotina de criação de ficheiros de dados}

A seguir explicamos o procedimento
para se extrair um novo mapa de
\url{https://www.openstreetmap.org/}
e incluí-lo no nosso projeto.

Para exportar um mapa de \url{https://www.openstreetmap.org/}
delimitamos a área pretendida explicitamente
(\external{Export > Manually select a different area})
e pedimos o download, preferencialmente usando a Overpass API
(já que o ficheiro exportado diretamente a partir de
\external{Export} gera mapas muito incompletos).
Depois corremos o ficheiro na aplicação \exe{OpenStreetMapParser},
fornecida no moodle, com o ficheiro exportado.

O programa gera três ficheiros: um contendo a informação
relativa aos \file{nodes} -- os vértices do grafo, que são os
locais onde há interseções de ruas, ou então curvas apertadas,
permitindo que uma rua curvilínea seja decomposta numa sequência
de vértices e arestas -- outro contendo informação relativa às
\file{roads} e outro contendo informação relativa às
\file{subroads} -- as arestas do grafo, que geram a rede de
ruas e estradas entre as interseções (alguns dos \file{nodes})
e definem os vértices atravessados por cada \file{road}.

Além destes três ficheiros, um quarto ficheiro \file{meta}
deve ser criado com a informação geral do mapa.
Este ficheiro deverá ter os seguintes campos, em estilo key=val;

\begin{center}
\begin{tabular}{rl}
\textbf{min\_longitude} &A longitude mínima do grafo.\\
\textbf{max\_longitude} &A longitude máxima do grafo.\\
\textbf{min\_latitude} &A latitude mínima do grafo.\\
\textbf{max\_latitude} &A latitude máxima do grafo.\\
\textbf{nodes} &O número de nodes (do mapa e do grafo).\\
\textbf{edges} &O número de edges (do mapa e do grafo).\\
\textbf{density} &(Opcional) A densidade de vértices no grafo.\\
\textbf{width} &(Opcional) A largura pretendida do grafo.\\
\textbf{height} &(Opcional) A altura pretendida do grafo.\\
\end{tabular}
\end{center}

Os quatro limites geográficos são, naturalmente, decimais que
delimitam o mapa. Todos os nodes armazenados no ficheiro de
\file{nodes} devem estar enquadrados na zona geograficamente
delimitada por estes limites. Os campos nodes e edges são
inteiros e devem indicar o número de vértices e arestas
guardados nos ficheiros \file{nodes} e \file{subroads}
respetivamente. Todos estes parâmetros obrigatórios são
fornecidos pelo \exe{OpenStreetMapParser} (mas as coordenadas
não têm de ser necessariamente essas).

Os ficheiros deverão ser colocados na pasta \folder{resources}.
A convenção de nomenclatura dos ficheiros é simples. Escolhido
um nome representativo qualquer, por exemplo \file{city},
os quatro ficheiros deverão ser chamados:

\begin{center}
\begin{tabular}{rl}
\textbf{\file{city_meta.txt}} &O ficheiro \file{meta} (manualmente criado).\\
\textbf{\file{city_nodes.txt}} &O ficheiro \file{nodes} (o primeiro gerado).\\
\textbf{\file{city_roads.txt}} &O ficheiro \file{roads} (o segundo gerado).\\
\textbf{\file{city_subroads.txt}} &O ficheiro \file{subroads} (o terceiro gerado).\\
\end{tabular}
\end{center}
Esta nomenclatura será a convenção o resto do relatório.



\subsection{Cálculo das dimensões do grafo}

Os vértices listados no ficheiro \file{nodes} têm a sua localização
apresentada em coordenadas geográficas. Para guardar os vértices no
\class{GraphViewer} estas coordenadas têm de ser transformadas
em coordenadas cartesianas (X,Y) de forma a manter-se a
proporcionalidade do mapa.

Este problema resolvemos em duas etapas: primeiro calculamos
as distâncias reais em quilómetros (segundo geodésicas horizontais
e verticais) com as fórmulas

\begin{gather*}
\delta\sfmath{latitude}=\frac{\sfmath{max_latitude}-\sfmath{min_latitude}}{2}\\
\delta\sfmath{longitude}=\frac{\sfmath{max_longitude}-\sfmath{min_longitude}}{2}\\
\overline{\sfmath{latitude}}=\frac{\sfmath{min_latitude}+\sfmath{max_latitude}}{2}\\\\
\sfmath{distancia Y (km)}=110.574\cdot\delta\sfmath{latitude}\\
\sfmath{distancia X (km)}=111.320\cdot\delta\sfmath{longitude}\cdot
|\cos(\overline{\sfmath{latitude}})|\\
\end{gather*}

Estas fórmulas dão-nos a razão \sf{altura:largura}, que podemos
usar para inicializar o \class{GraphViewer} com as proporções
corretas. A segunda etapa é escolher um valor conveniente ou para
\sf{largura} ou para \sf{altura}. Basta escolher uma coordenada
para o Grafo, e a outra coordenada é devidamente calculada
usando a razão acima.

Esta largura base ou altura base pode ser especificada como parâmetro
opcional no ficheiro \file{meta} (se ambas forem especificadas são
ignoradas).

A alternativa é indicar qual deverá ser a densidade dos vértices
no grafo. A densidade é vista da seguinte forma: se o grafo
tem $V$~vértices, então a densidade dos vértices no grafo é
\begin{equation*}
\frac{V}{\sfmath{largura}\cdot\sfmath{altura}}
\end{equation*}
e se quisermos que a densidade seja $d$ então

\begin{equation*}
d=\frac{V}{\sfmath{largura}\cdot\sfmath{altura}}
\Rightarrow
\sfmath{largura}=\sqrt{\frac{V}{d\cdot\sfmath{altura:largura}}}
\end{equation*}
e podemos calcular a \sf{largura} e a \sf{altura} do grafo
com a devida densidade~$d$.

Em caso de omissão dos parâmetros opcionais é usada densidade $0.0001$.



\subsection{Leitura dos ficheiros de dados}

Todos os ficheiros de dados são lidos usando classes da STL,
nomeadamente \sf{std::ifstream} e expressões regulares
\sf{std::regex}. Isto permite que erros nos ficheiros
de dados sejam detetados de forma simples e direta:

\begin{itemize}
\item Os parâmetros obrigatórios do ficheiro \file{meta} têm
de estar todos presentes.
\item O número de vértices indicado em \file{meta} deve ser
igual ao número de nodes (linhas) lidos em \file{nodes}.
\item O número de arestas indicado em \file{meta} deve ser
igual ao número de subroads (linhas) lidas em \file{subroads}.
\item As latitudes e as longitudes lidas têm de estar dentro
dos limites especificados.
\item \textellipsis
\end{itemize}

As expressões regulares usadas nas funções de leitura são:

\begin{center}
\begin{tabular}{rll}
\file{meta} &\sf{attr=val;}\\
&\verb@\attr ?= ?(-?\d+\.?\d*)[.;,]\i@  \qquad (decimais)\\
&\verb@\attr ?= ?(\d+)[.;,]\i@          \qquad (ints)\\
\\
\file{nodes} &\sf{node_id;lat_deg;long_deg;long_rad;lat_rad}\\
&\verb@\^(\d+);(-?\d+.?\d*);(-?\d+.?\d*);(?:-?\d+.?\d*);(?:-?\d+.?\d*);?$\@\\
\\
\file{roads} &\sf{road_id;road_name;two_way}\\
&\verb@\^(\d+);((?:[-0-9a-zA-ZÀ-ÿ,\.]| )*);(False|True);?$\@\\
\\
\file{subroads} &\sf{road_id;node1_id;node2_id;}\\
&\verb@\^(\d+);(\d+);(\d+);?$\@\\
\end{tabular}
\end{center}

Sequencialmente, lê-se a informação do ficheiro \file{meta};
inicializa-se o \class{GraphViewer}; lê-se o ficheiro
\file{nodes} e criam-se os vértices; lê-se o ficheiro
\file{roads} e criam-se as ruas; lê-se o ficheiro
\file{subroads} e criam-se as arestas.
\end{document}